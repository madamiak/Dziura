% Beamer
\PassOptionsToPackage{table}{xcolor}
\documentclass[unicode,14pt]{beamer}
\usepackage[utf8]{inputenc}
\usepackage{lmodern}
\usepackage[T1]{fontenc}
\usepackage{textcomp}
\usepackage{multirow}
\usepackage{rotating}

% Styl Warsaw i kolory wolvering -- taki mi się spodobał -- oczywiście można zmienić na inny
\usetheme{AnnArbor}
\usecolortheme{crane}

% Zmieniony pasek nawigacji
\setbeamertemplate{navigation symbols}{
  \insertslidenavigationsymbol   % to są slajdy
  \insertframenavigationsymbol   % następnie \frame
  \insertsectionnavigationsymbol % \section
  \insertdocnavigationsymbol     % cały dokument (na początek / na koniec)
}

% Żeby było widać ukryty tekst
\setbeamercovered{transparent}

% Tabele wyglądają lepiej
\renewcommand\arraystretch{1.2}

% Kolorowanie tabel
\rowcolors[]{1}{blue!20}{blue!10}

% Dane wstawiane na stronę tytułową i do info PDF
\title[System do zgłaszania uszkodzeń w mieście]{\emph{Dziura} -- system do zgłaszania \\ uszkodzeń w mieście}
\author{Grupa \emph{Alpha}}
\date{27 marca 2012}
\institute[PWr]{Politechnika Wrocławska}

\begin{document}

% Slajd ze stroną tytułową

\begin{frame}
  \titlepage
\end{frame}

% Slajd ze spisem treści

\begin{frame}

  \frametitle{Spis treści}
  \setcounter{tocdepth}{1}
  \tableofcontents

\end{frame}

% Na początku \section będzie powtórzony spis treści
\AtBeginSection[]
{
  \begin{frame}<beamer>
    \setcounter{tocdepth}{1}
    \tableofcontents[currentsection]
  \end{frame}
}

% Włączenie poszczególnych części prezentacji:
% Na razie przykładowo:
\section{Wstęp}

\subsection{Cel projektu}

\begin{frame}

\textbf{Celem projektu jest:}
\begin{itemize}
  \item A
  \item B
  \item C
\end{itemize}

\end{frame}

%------------------------------------

\subsection{Kolejny slajd}

\begin{frame}

\textbf{Cośtam:}

\begin{itemize}
  \item 1
  \item 2
  \item 3
\end{itemize}

\end{frame}

\section{Technologia}

\begin{frame}

bla bla bla

\end{frame}
\section{Aplikacja webowa}

\begin{frame}

bla bla bla

\end{frame}
\section{Aplikacja mobilna}

\begin{frame}

bla bla bla

\end{frame}
\begin{frame}

\begin{center}

\vfill
\mbox{}

{\textbf{Koniec.} }

\vspace{0.5cm}

{ \textbf{Pytania?} }

\vspace{1cm}

\begin{minipage}{0.8\textwidth}
\footnotesize
tutaj jakiś kontakt do nas \\
\end{minipage}

\vfill
\mbox{}

\end{center}

{\tiny
\begin{thebibliography}{9}
  \bibitem{przyklad} {Bibliografię też można jakąś podać}
\end{thebibliography}
}

\end{frame}


\end{document}
